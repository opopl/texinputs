\documentclass[a4paper,11pt]{report}

\usepackage[letterpaper,unicode,linktocpage,bookmarksdepth=subparagraph]{hyperref}
\usepackage{appendix}
%\usepackage{bookmark}
\usepackage{mathtext}                 % підключення кирилиці у математичних формулах
                                          % (mathtext.sty входить в пакет t2).
\usepackage[OT1,T2A,T3]{fontenc}         % внутрішнє кодування шрифтів (може бути декілька);
                                          % вказане останнім діє по замовчуванню;
                                          % кириличне має співпадати з заданим в ukrhyph.tex.
\usepackage[utf8]{inputenc}       % кодування документа; замість cp866nav
                                          % може бути cp1251, koi8-u, macukr, iso88595, utf8.
\usepackage[english]{babel} % національна локалізація; може бути декілька
                                          % мов; остання з переліку діє по замовчуванню. 
\usepackage{url}
\usepackage{csquotes}
\usepackage{color}
\usepackage{makeidx}
\usepackage{my}
\usepackage{multicol}
\usepackage[text={7in,9in},centering]{geometry}
\usepackage{longtable}
\usepackage{tipa}
\usepackage{graphicx}
%%end_usepackage

\makeatletter
\renewcommand\paragraph{%
   \@startsection{paragraph}{4}{0mm}%
      {-\baselineskip}%
      {.5\baselineskip}%
      {\normalfont\normalsize\bfseries}}
\makeatother

\newcommand{\ccdash}{"}
\newcommand{\defis}{-}

\def\booktitle#1{\textbf{#1}}

