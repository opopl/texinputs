
\usepackage{listings}

\lstset{% 
	language=Perl,                	% the language of the code
	backgroundcolor=\color{YellowOrange},
	escapeinside={(*@}{@*)},
	basicstyle=\normalfont\ttfamily\small,       % the size of the fonts that are used for the code
	numbers=left,                   % where to put the line-numbers
	numberstyle=\small\color{gray},  % the style that is used for the line-numbers
	stepnumber=2,                   % the step between two line-numbers. If it's 1, each line 
																	% will be numbered
	numbersep=5pt,                  % how far the line-numbers are from the code
	showspaces=false,               % show spaces adding particular underscores
	showstringspaces=false,         % underline spaces within strings
	showtabs=false,                 % show tabs within strings adding particular underscores
	frame=single,                   % adds a frame around the code
	rulecolor=\color{black},        % if not set, the frame-color may be changed on line-breaks within not-black text (e.g. commens (green here))
	tabsize=2,                      % sets default tabsize to 2 spaces
	captionpos=b,                   % sets the caption-position to bottom
	breaklines=true,                % sets automatic line breaking
	breakatwhitespace=false,        % sets if automatic breaks should only happen at whitespace
	title=\lstname,                   % show the filename of files included with \lstinputlisting;
																	% also try caption instead of title
	keywordstyle=\color{Red},          % keyword style
	commentstyle=\color{Blue},       % comment style
	stringstyle=\color{mauve},         % string literal style
	escapeinside={\%*}{*)},            % if you want to add a comment within your code
	morekeywords={*,...}               % if you want to add more keywords to the set
}%%

%\lstnewenvironment{Perl}{
%}{}
%  \lstset{%
    %language={perl},
    %basicstyle=\normalfont\ttfamily\footnotesize,       
    %keywordstyle=\color{black},         % style for keyword
    %emph={% From http://www.sdsc.edu/~moreland/courses/IntroPerl/docs/manual/pod/perlfunc.html
    %-X, run, abs, absolute, accept, accept, alarm, schedule, atan2, 
    %arctangent, bind, binds, binmode, prepare, bless, create, caller, 
    %get, chdir, change, chmod, changes, chomp, remove, chop, remove, 
    %chown, change, chr, get, chroot, make, close, close, closedir, close, 
    %connect, connect, continue, optional, cos, cosine, crypt, one-way, 
    %dbmclose, breaks, dbmopen, create, defined, test, delete, deletes, 
    %die, raise, do, turn, dump, create, each, retrieve, endgrent, be, 
    %endhostent, be, endnetent, be, endprotoent, be, endpwent, be, 
    %endservent, be, eof, test, eval, catch, exec, abandon, exists, test, 
    %exit, terminate, exp, raise, fcntl, file, fileno, return, flock, 
    %lock, fork, create, format, declare, formline, internal, getc, get, 
    %getgrent, get, getgrgid, get, getgrnam, get, gethostbyaddr, get, 
    %gethostbyname, get, gethostent, get, getlogin, return, getnetbyaddr, 
    %get, getnetbyname, get, getnetent, get, getpeername, find, getpgrp, 
    %get, getppid, get, getpriority, get, getprotobyname, get, 
    %getprotobynumber, get, getprotoent, get, getpwent, get, getpwnam, 
    %get, getpwuid, get, getservbyname, get, getservbyport, get, 
    %getservent, get, getsockname, retrieve, getsockopt, get, glob, 
    %expand, gmtime, convert, goto, create, grep, locate, hex, convert, 
    %import, patch, int, get, ioctl, system-dependent, join, join, keys, 
    %retrieve, kill, send, last, exit, lc, return, lcfirst, return, 
    %length, return, link, create, listen, register, local, create, 
    %localtime, convert, log, retrieve, lstat, stat, m//, match, map, 
    %apply, mkdir, create, msgctl, SysV, msgget, get, msgrcv, receive, 
    %msgsnd, send, my, declare, next, iterate, no, unimport, oct, convert, 
    %open, open, opendir, open, ord, find, pack, convert, package, 
    %declare, pipe, open, pop, remove, pos, find, print, output, printf, 
    %output, prototype, get, push, append, q/STRING/, singly, qq/STRING/, 
    %doubly, quotemeta, quote, qw/STRING/, quote, qx/STRING/, backquote, 
    %rand, retrieve, read, fixed-length, readdir, get, readlink, 
    %determine, recv, receive, redo, start, ref, find, rename, change, 
    %require, load, reset, clear, return, get, reverse, flip, rewinddir, 
    %reset, rindex, right-to-left, rmdir, remove, s///, replace, scalar, 
    %force, seek, reposition, seekdir, reposition, select, reset, semctl, 
    %SysV, semget, get, semop, SysV, send, send, setgrent, prepare, 
    %sethostent, prepare, setnetent, prepare, setpgrp, set, setpriority, 
    %set, setprotoent, prepare, setpwent, prepare, setservent, prepare, 
    %setsockopt, set, shift, remove, shmctl, SysV, shmget, get, shmread, 
    %read, shmwrite, write, shutdown, close, sin, return, sleep, block, 
    %socket, create, socketpair, create, sort, sort, splice, add, split, 
    %split, sprintf, formatted, sqrt, square, srand, seed, stat, get, 
    %study, optimize, sub, declare, substr, get, symlink, create, syscall, 
    %execute, sysread, fixed-length, system, run, syswrite, fixed-length, 
    %tell, get, telldir, get, tie, bind, time, return, times, return, 
    %tr///, transliterate, truncate, shorten, uc, return, ucfirst, return, 
    %umask, set, undef, remove, unlink, remove, unpack, convert, unshift, 
    %prepend, untie, break, use, load, utime, set, values, return, vec, 
    %test, wait, wait, waitpid, wait, wantarray, get, warn, print, write, 
    %print, y///, transliterate},       
		%% define a list of word to emphasis
    %stringstyle=\color{red},
    %emphstyle=\color{black}\bfseries,   % define the way to emphase
    %showspaces=false,                   % show the space in code, or not
    %stringstyle=\ttfamily,              % style of the string (like "hello word")
    %showstringspaces=false,             % show the space in string, on not #1
    %commentstyle=\color{gray}\slshape,
    %tabsize=2,                          % sets default tabsize to 2 spaces
    %breaklines=true,                    % sets automatic line breaking
    %breakatwhitespace=false,            % sets if automatic breaks should only happen at whitespace
  %}

%}{}

%\lstset{language=Perl}


%\lstset{
    %basicstyle=\ttfamily,
    %language=Perl,
    %keywordstyle=\rmfamily\bfseries,
    %commentstyle=\sffamily,
%}
%\def\mystyle{}


